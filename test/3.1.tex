\chapter{参考文献标注法}

\section{顺序编码制}

\gbtbibstyle{numerical}

……改变社会规范也可能存在类似的“二阶囚徒困境”问题:尽管改变旧的规范对所有人都好,
但个人理性选择使得没有人愿意率先违反旧的规范\cite{sunstein}。
……事实上,古希腊对轴心时代思想真正的贡献不是来自对民主的赞扬,而是来自对民主制度
的批评,苏格拉底、柏拉图和亚里士多德3位贤圣都是民主制度的坚决反对者
\cite[20]{morri}。
……柏拉图在西方世界的影响力是如此之大以至于有学者评论说,一切后世的思想都是一系列
为柏拉图思想所作的脚注\cite{ljs}。
……据《唐会要》记载,当时拆毁的寺院有4 600余所,招提、兰若等佛教建筑4万余所,没收
寺产,并强迫僧尼还俗达260 500人。佛教受到极大的打击\cite[326-329]{morri}。
……陈登原先生的考证是非常精确的,他印证了《春秋说题辞》“黍者绪也,故其立字,禾入
米为黍,为酒以扶老,为酒以序尊卑,禾为柔物,亦宜养老”,指出:“以上谓等威之辨,尊
卑之序,由于饮食荣辱。”\cite{cdy}

\bibliography{test}
